% !TEX TS-program = pdflatex
\documentclass[11pt,a4paper]{article}
\usepackage[utf8]{inputenc}
\usepackage[T1]{fontenc}
\usepackage{lmodern}
\usepackage{geometry}
\geometry{margin=1in}
\usepackage{amsmath,amssymb,amsthm,mathtools}
\usepackage{enumitem}
\usepackage{hyperref}
\usepackage{microtype}
\usepackage{cleveref}
% \setlist[itemize]{noitemsep,topsep=2pt}
% \setlist[enumerate]{noitemsep,topsep=4pt,label=\arabic*.}


  \newtheorem{theorem}{Theorem}
  \newtheorem{definition}{Definition}

  \title{Project Description for Optimization~1: Location Analysis}
  \author{}
  \date{February 9, 2026}

  \begin{document}
  \maketitle\footnotetext{This project is based on the 2022 project
    written by Elizabeth K\"obis.}

  \section*{Submission}

  You are strongly encouraged to work on this project in groups of up
  to three. Each member of the group must submit an (identical) copy
  of the project before the deadline 23:59 Sunday 22.3.2026 on
  Inspera.

  All members are expected to contribute on, and should be ready to
  answers questions to, all parts of the project. \emph{For full
    marks, all numbered points should be addressed.}

  The project should be completed as a written report, including a
  typeset \verb|pdf| file. Background, calculations, justifications
  and conclusions should be clearly stated. All group members names
  must be displayed clearly on the \verb|pdf| file.

  The coding component must be runnable. Please submit your code as a
  Jupyter notebook with minimal external dependencies. In Inspera, the
  pdf is to be delivered separately, and the code as a \verb|zip|
  file. The clarity of your report counts towards the final grade.

  Please direct questions about the project towards Anja Ringstad, who
  can be reached via email at \texttt{anjarin@stud.ntnu.no}.


  \section{Introduction}
  Many important problems from industry, engineering-technical or social
  area, logistics and the energy industry lead to location
  problems. Examples of this are the planning of new industrial plants,
  production facilities or warehouses, of hospitals, rescue stations and
  other public facilities, to landscaping or the placement of sensors on
  technical components. For example, in order to avoid hospital spatial
  inefficiency, it is important to learn how to optimize the spaces such
  that relevant services such as surgical equipment or emergency
  services are located at the best points of access to other
  stations. As another example, for a decision-maker deciding the best
  location to build a warehouse relative to factories, it is important
  to find the best place to locate this warehouse in order to minimize
  costs of travel and transportations. Thus, location problems appear in
  many variants and with different constraints depending on the practical
  application. Other examples of location analysis appear for instance
  in the following areas:
  \begin{itemize}

  \item Urban and regional planning (e.g., locations for emergency
    facilities)

  \item Technology (e.g., placement of sensors on technical components)

  \item Economy (e.g., planning new production facilities)

  \item Geography (e.g., landscape design)

  \item Environment-oriented project management (e.g., development of
    mining landscapes)

  \item Engineering.

  \end{itemize}

  This project is intended to provide the mathematical tools to model
  and solve location problems.  The main goal in location analysis is to
  find one or more new locations with minimal distance to a set of known
  locations.

  \section{Problem description}

  Let us consider $m$ given points in the plane

  $$
  a^i=\left(a_1^i, a_2^i\right)^{\top} \in \mathbb{R}^2, \quad i=1, \ldots, m .
  $$

  $A=\left\{a^1, \ldots, a^m\right\}$ is the set of all given
  locations. Let $\mathcal{M}:=\{1, \ldots, m\}$. We denote the new
  location, which we want to determine, by

  $$
  x=\left(x_1, x_2\right)^{\top} \in \mathbb{R}^2 .
  $$


  When modeling a location problem, distance measures as well as the
  selected objective function play a key role. We consider the following
  distances:
  \begin{itemize}
  \item $d_1(x, y)=\left|x_1-y_1\right|+\left|x_2-y_2\right|$ (Manhattan
    or Taxicab distance)
  \item $d_2(x, y)=\sqrt{\left(x_1-y_1\right)^2+\left(x_2-y_2\right)^2}$
  (Euclidean distance)
\item $d_{\infty}(x, y)=\max
  \left\{\left|x_1-y_1\right|,\left|x_2-y_2\right|\right\}$ (Maximum
  distance)
\end{itemize}

The individual distances between the unknown $x$ and a given $a^i$ are
set differently according to the concrete application. The selection
of the distance measure determines whether the distance should be
measured by the linear distance (Euclidean) or by a blocknorm (Maximum
or Manhattan). The Maximum distance is suitable if movement in both
directions is possible and only the larger one of the two determines
the length of the movement.

For the objective function, the following ones are possible:
\begin{itemize}
\item $\max _{a \in A} d(a, x)$ (Center problem)
\item  $\sum_{a \in A} d(a, x)$ (Median function)
\end{itemize}

Thus, we consider the following problems

\begin{equation} \label{2.1}
  \min _{x \in \mathbb{R}^2} \max _{a \in A} d(a, x) \text { (Center
    problem) }
\end{equation}
\begin{equation} \label{2.2}
  \min _{x \in \mathbb{R}^2} \sum_{a \in A} d(a, x) \text { (Median
    problem, also
    called
    Fermat-Weber
    problem) }
\end{equation}

\subsection{Metrics and Norms}

Mathematically, the distances between two points in the plane can be
measured using an appropriate metric.
\begin{definition}
  Let $Y$ be a
  non-empty set of $\mathbb{R}^2$. A function
  $d: Y \times Y \rightarrow \mathbb{R}$ is called metric on $Y$ if $d$
  fulfills the following conditions for all $x, y, z, \in Y$ :
  \begin{itemize}
  \item $d(x, y)=0 \Longleftrightarrow x=y$ (definiteness).
  \item $d(x, y)=d(y, x)$ (symmetry).
  \item $d(x, z) \leq d(x, y)+d(y, z)$
    (triangle inequality).
  \end{itemize}

  Note that $d(x, y) \in\mathbb{R}$ represents the distance between the points
  $x$ and $y$.
\end{definition}

\begin{definition}
  A function $\|\cdot\|: \mathbb{R}^2 \rightarrow \mathbb{R}$ is
  called norm on $\mathbb{R}^2$ if $\|\cdot\|$ fulfills the following
  conditions for all $x, y \in \mathbb{R}^2$ and for all $\alpha \in
  \mathbb{R}$ :
  \begin{itemize}
  \item  $\|x\|=0 \Longleftrightarrow x=0$ (definiteness).
  \item $\|\alpha x\|=|\alpha| \cdot\|x\|$ (positive homogeneity).
  \item  $\|x+y\| \leq\|x\|+\|y\|$ (triangle inequality).
  \end{itemize}
    
  If $\|\cdot\|: \mathbb{R}^2 \rightarrow \mathbb{R}$ is a norm, then
  it is possible to define a metric on $\mathbb{R}^2$ that is induced
  by the norm $\|\cdot\|$ in the following way:

  $$
  d(x, y):=\|x-y\|
  $$

  for all $x, y \in \mathbb{R}^2$.
\end{definition}

\subsection{Project Tasks - Theoretical Work}
\begin{enumerate}
\item Show that $d_1, d_2$ and $d_{\infty}$ are metrics.
\item  Plot the unit balls $B_i(0,1):=\left\{x \in \mathbb{R}^2 \mid\|x\|_i \leq 1\right\}(i \in\{1,2, \infty\})$ of the so-called Manhattan norm, the Euclidean norm and of the maximum norm.
\item  Show that every norm is a convex function.
\item  Show that the objective function of problems \eqref{2.1} and
  \eqref{2.2} is convex. 
\item  Give a geometric interpretation of solving problem \eqref{2.1} with Euclidean distance function.
\item  Provide a solution approach for problem \eqref{2.2} with Manhattan
  distance function (Hint: Note that the objective function can be
  separated (so-called separability). Just describe the idea of such
  an approach.).
\item Consider the Median problem with squared Euclidean distance function

  \[
  \min _{x \in \mathbb{R}^{2}} \sum_{a \in A}\left(d_{2}(a, x)\right)^{2} .
\]

Show that the objective function is convex, and analytically compute
the uniquely determined minimizer of this problem. [10 points]

\item Consider the Median problem with Euclidean distance function

  \[
  \min _{x \in \mathbb{R}^{2}} \sum_{a \in A} d_{2}(a, x) .
\]

Derive necessary optimality conditions for minimizers of this problem
(Hint: Note that the partial derivatives are not defined globally).

\item
  Give an example of a set of existing locations, where the set of
  global minimizers for the problem \(\min _{x \in \mathbb{R}^{2}}
  \sum_{a \in A} d_{2}(a, x)\) and \(\min _{x \in \mathbb{R}^{2}}
  \sum_{a \in A}\left(d_{2}(a, x)\right)^{2}\) do not
  coincide. Explain the solution.
\end{enumerate}

\section{Weiszfeld Algorithm}
The above problem in task 7. was easy to solve, as the objective function is differentiable and one is able to derive the stationary points by rearranging the formula appearing in the optimality condition. If we consider the Median problem with Euclidean distance function that is not squared, an according procedure is not as simple anymore, as one cannot solve for a minimizer directly. Instead, one uses an iterative procedure to compute approximations of a minimizer. Here, we consider the Median problem with weighted Euclidean distances as follows

\begin{equation}\label{P}
  \min _{x \in \mathbb{R}^{2}} \sum_{i \in \mathcal{M}} v^{i} d_{2}\left(a^{i}, x\right)
  \end{equation}

  where the weights \(v^{i}, i \in \mathcal{M}=\{1, \ldots, m\}\), are real-valued nonnegative numbers.\\
  The problem \eqref{P} is one of the most fundamental location problems. It consists of finding a point that minimizes the sum of its weighted distances to a given finite set of anchor points. The problem is credited to the well-known French mathematician Pierre de Fermat, who at the beginning of the seventeenth century posed the following question:

  \begin{center}
    \emph{Given three points in a plane, find a fourth point such that
      the sum of its distances to the three given points is as small
      as possible.}
  \end{center}

  The Italian physicist and mathematician Evangelista Torricelli (mostly known for inventing the barometer) found a construction method of this point by ruler and compass, and it is therefore also called "the Toricelli point" (see reference 2. in Section 4). At the beginning of the twentieth century, the German economist Alfred Weber incorporated weights, and was able to treat facility location problems with more than 3 facilities, and the problem was consequently called "the Fermat-Weber problem". Other names for the problem are "the Fermat problem", "the Weber problem", "the Fermat-Toricelli problem", "the Steiner problem", and many more variants.

  Before presenting an algorithm for solving problem \eqref{P}, we give the following result.\\
  
  \begin{theorem}(see: Love, Morris Wesolowsky, 1988, Property 2.2). \label{thm1}
  If for
  \[
      \text { Test }_{k}:=\left[\left(\sum_{i \in \mathcal{M} \backslash\{k\}} v^{i} \frac{a_{1}^{k}-a_{1}^{i}}{d_{2}\left(a^{k}, a^{i}\right)}\right)^{2}+\left(\sum_{i \in \mathcal{M} \backslash\{k\}} v^{i} \frac{a_{2}^{k}-a_{2}^{i}}{d_{2}\left(a^{k}, a^{i}\right)}\right)^{2}\right]^{\frac{1}{2}},
  \]
  it holds that
  \[
  \operatorname{Test}_{k} \leq v^{k}
  \]
  for one \(k \in \mathcal{M}\), then \(x^{*}=a^{k}\) is a global minimizer.
  \end{theorem}

The so-called Weiszfeld algorithm is an iterative method based on the first-order necessary conditions for a stationary point of the objective function.

The Weiszfeld algorithm consists of the following steps:

\subsubsection*{Weiszfeld algorithm for solving \eqref{P}}
Input. Existing locations \(a^{i}=\left(a_{1}^{i}, a_{2}^{i}\right)^{T} \in \mathbb{R}^{2}\) (that are pairwise different from each other), weights \(v^{i}>0, i \in \{1, \ldots, m\}\), error bound \(\epsilon>0\).

\begin{enumerate}[label=(\alph*)]
\item Check if the minimum is attained at an existing location
  \(a^{i}, i \in\{1, \ldots, m\}\). If for
  \[
    \operatorname{Test}_{k}:=\left[\left(\sum_{i \in \mathcal{M}
          \backslash\{k\}} v^{i}
        \frac{a_{1}^{k}-a_{1}^{i}}{d_{2}\left(a^{k},
            a^{i}\right)}\right)^{2}+\left(\sum_{i \in \mathcal{M}
          \backslash\{k\}} v^{i}
        \frac{a_{2}^{k}-a_{2}^{i}}{d_{2}\left(a^{k},
            a^{i}\right)}\right)^{2}\right]^{\frac{1}{2}},
  \]
  it holds that
  \[
    \operatorname{Test}_{k} \leq v^{k}
  \]
  for one \(k \in \mathcal{M} \Longrightarrow\) Set \(x^{*}=a^{k}\) and
  END. Otherwise go to Step (b).

\item Choose a starting point \(x=\left(x_{1}, x_{2}\right)\). This point can be found, for example, by solving the median problem using the squared Euclidean norm.
\item  Set for \(j=1,2: x_{j}^{\text {new }}=\frac{\sum_{i=1}^{m} v^{i} \frac{a_{j}^{i}}{d_{2}\left(a^{i}, x\right)}}{\sum_{i=1}^{m} v^{i} \frac{1}{d_{2}\left(a^{i}, x\right)}}\).\\
\item If \(x^{\text {new }}=\left(x_{1}^{\text {new }}, x_{2}^{\text {new }}\right)\) satisfies a stopping criterion w.r.t. the given \(\epsilon\) (see \S\ref{sec:terminate} below) then set \(x^{*}=x^{\text {new }}\). Otherwise: Set \(x=x^{\text {new }}\) and go to Step (c).

\end{enumerate}
Output. Approximation of the minimizer \(x^{*}=\left(x_{1}^{*}, x_{2}^{*}\right)\).\\
The main iteration scheme in the algorithm is

\begin{equation} \label{iterscheme}
  \begin{split}
    x_{1}^{\text {new }} & :=\frac{\sum_{i=1}^{m} v^{i} \frac{a_{1}^{i}}{d_{2}\left(a^{i}, x\right)}}{\sum_{i=1}^{m} v^{i} \frac{1}{d_{2}\left(a^{i}, x\right)}} \\
    x_{2}^{\text {new }} & :=\frac{\sum_{i=1}^{m} v^{i}
                           \frac{a_{2}^{i}}{d_{2}\left(a^{i},
                           x\right)}}{\sum_{i=1}^{m} v^{i}
                           \frac{1}{d_{2}\left(a^{i}, x\right)}}
  \end{split}
\end{equation}

\subsection{Termination Criterion} \label{sec:terminate}
Let us define \(f_{d_{2}}(x):=\sum_{i \in \mathcal{M}} v^{i} d_{2}\left(a^{i}, x\right)\). Let \(x \in \mathbb{R}^{2}\) be an iteration point that was obtained during the Weiszfeld algorithm. We define

\[
  \sigma(x):=\max \left\{d_{2}(x, y): y \in \operatorname{conv}\left\{a^{1}, \ldots, a^{m}\right\}\right\}
\]

with \(\operatorname{conv}\left\{a^{1}, \ldots, a^{m}\right\}:=\left\{x \in \mathbb{R}^{2}: \exists \lambda^{1} \geq 0, \ldots, \lambda^{m} \geq 0, \sum_{i=1}^{m} \lambda^{i}=1, x=\sum_{i=1}^{m} \lambda^{i} a^{i}\right\}\) being the convex hull.\\

\begin{theorem} \label{thm2}
  Let \(x \in \mathbb{R}^{2}\) be an iteration point that was obtained during the Weiszfeld algorithm. Then
  \[
      U B(x):=\left\|\nabla f_{d_{2}}(x)\right\| \cdot \sigma(x)
  \]
  is an upper bound for a possible improvement of the objective function
  \(f_{d_{2}}(x)\), where \(\|\cdot\|\) is the Euclidean norm in
  \(\mathbb{R}^{2}\).
\end{theorem}


\begin{proof}
Because \(f_{d_{2}}\) is convex, we have for an iteration point \(x\) obtained during the Weiszfeld algorithm
\[
  \begin{aligned}
    f_{d_{2}}(x)-f_{d_{2}}\left(x^{*}\right) & \leq\left\langle\nabla f_{d_{2}}(x), x-x^{*}\right\rangle \\
                                             & \overset{Cauchy-Schwarz}{\leq} \left\|\nabla f_{d_{2}}(x)\right\| \cdot\left\|x-x^{*}\right\| \\
                                             & \leq\left\|\nabla f_{d_{2}}(x)\right\| \cdot \sigma(x)
  \end{aligned}
\]
\end{proof}

\begin{theorem} \label{thm3}
  Let \(x^{*}\) denote the global minimizer of \eqref{P}. A lower bound for \(f_{d_{2}}\left(x^{*}\right)\) is given by
  \[
    L B(x):=f_{d_{2}}(x)-\left\|\nabla f_{d_{2}}(x)\right\| \cdot \sigma(x) .
  \]
\end{theorem}

\begin{proof}
  We have from Theorem 2

  \[
  f_{d_{2}}(x)-f_{d_{2}}\left(x^{*}\right) \leq\left\|\nabla f_{d_{2}}(x)\right\| \cdot \sigma(x),
\]

and therefore

\[
  f_{d_{2}}(x)-\left\|\nabla f_{d_{2}}(x)\right\| \cdot \sigma(x) \leq f_{d_{2}}\left(x^{*}\right) .
\]
\end{proof}
Based on these two results, show that the following consequence holds:\\
\begin{theorem} \label{thm4}
  If for the lower bound, we have \(L B(x)>0\) and for a given \(\epsilon>0\)

  \[
      \frac{U B(x)}{L B(x)}=\frac{\left\|\nabla f_{d_{2}}(x)\right\|
        \cdot \sigma(x)}{f_{d_{2}}(x)-\left\|\nabla
          f_{d_{2}}(x)\right\| \cdot \sigma(x)}<\epsilon
    \]

    holds true, then the current iterate has at least a relative
    accuracy\footnote{\({ }^{1}\) The relative accuracy of the current
      iterate \(x\) is defined as
      \(\left|\frac{f_{d_{2}}(x)-f_{d_{2}}\left(x^{*}\right)}{f_{d_{2}}\left(x^{*}\right)}\right|\).
    } of \(\epsilon\).
  \end{theorem}

  \subsection*{3.2 Project Tasks - Weiszfeld Algorithm}
  \begin{enumerate}[start=10]
  \item Explain why the main iteration scheme is chosen according to \eqref{iterscheme}.
  \item Prove \Cref{thm4} above using \Cref{thm2} and \Cref{thm3}.
  \item Derive a termination criterion for the Weiszfeld algorithm based on \Cref{thm4}.
  \item Implement the Weiszfeld algorithm using Python. Test your implementation on the following cases 
  \begin{enumerate}[label=(\roman*)]
      \item A unit square $A = \{(-1, -1)^{\top} , (-1, 1)^{\top}, (1, -1)^{\top}, (1,1)^{\top}\}$ with equal weights $v^{i} = 1$ for $ i = 1, \dots, 4$. Verify that the algorithm converges to the origin $x^* = (0,0)^{\top}$. 
      \item Create an original test case. Discuss how the set of given locations $A =\left\{a^1, \ldots, a^m\right\}$ and weights $v^{i},\ i \in \mathcal{M}=\{1, \ldots, m\}$ affects the location of the minimizer. 
  \end{enumerate}
  Your code must be runnable.
  \item Implement the gradient descent method with backtracking and
    replace steps (b)-(d) in the above algorithm with it. Test your implementation on the cases from above. Compare with the Weiszfeld algorithm and discuss the difference between the two
    algorithms and their performance. Which algorithm would you
    suggest, and why?
  \end{enumerate}

  \section*{4 Suggested Reading Material / Additional Resources}
  \begin{enumerate}[label=\Roman*.]
  \item R. F. Love, J. G. Morris and G. O. Wesolowsky: Facility Location: Models and Methods, North Holland, New York, 1988. The relevant pages of this reference can be found here: \href{http://web.tecnico.ulisboa}{http://web.tecnico.ulisboa}. pt/mcasquilho/compute/\_scicomp/\_location/LoveMorrisWesolowsky.pdf
  \item A. Beck, S. Sabach, S. Weiszfeld's Method: Old and New Results. J Optim Theory Appl 164, 1-40, 2015. \href{https://doi.org/10.1007/s10957-014-0586-7}{https://doi.org/10.1007/s10957-014-0586-7}
  \item Z. Drezner, H. W. Hamacher (Editors): Facility Location. Applications and Theory. Springer-Verlag Berlin Heidelberg, 2002.
  \item Reza Zanjirani Farahani, Masoud Hekmatfar (Editors): Facility Location: Concepts, Models, Algorithms and Case Studies. Springer Dordrecht Heidelberg London New York, 2009.
  \item \href{https://project-flo.de}{https://project-flo.de} (Project Facility Location Optimizer: A MATLAB-based software tool for solving location problems).
  \end{enumerate}




\end{document}
